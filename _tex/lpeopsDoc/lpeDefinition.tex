\chapter{LPE}

\section{Introduction}
The techniques that are described in the following chapters must be applied to a \emph{linear process equation}, or LPE for short.

\section{Requirements}

An LPE must satisfy the following requirements:

\begin{itemize}

\item Given are the definition of a single process \emph{and} its instantiation.
The instantiation must be a closed expression, and otherwise satisfy the usual requirements of \txs{} of process instantiations.
There are no requirements for the channel parameters and data parameters beyond the usual requirements of \txs{} of process definitions.

\item The body of the process consists of one or more \emph{summands}.
In this context, a summand must contain exactly the following, in the given order:

\begin{itemize}

\item One or more \emph{channel communications}; that is, channel references followed by the variables that are used to communicate over those channels.
For example, \inlinecode{OUTPUT ? x ? y}, or simply \inlinecode{OUTPUT}.

Channel communications are combined with \inlinecode{|}.
To give another example, \inlinecode{INPUT ? x | OUTPUT ? y}.

The variables that are used to communicate over a channel must be \emph{fresh}.

\item A guard.
Given the variables that are used by this summand to communicate over channels, the guard should be a closed expression that yields a boolean value.
The guard must also be \emph{satisfiable} (otherwise the summand would not have any effect anyway).
The guard is not always explicitly written if it is semantically equivalent to \inlinecode{true}.

\item A sequence operator, \inlinecode{>->}.

\item The expression for deadlock, \inlinecode{STOP}, \emph{or} a recursive process instantiation.
The instantiated process must use the process definition given by the LPE, and the channel parameters of the instantiated process must be assigned their current values of the instantiating process.
There are no requirements for the values assigned to the data parameters beyond the usual requirements of \txs{} (such as type compatibility).

\end{itemize}

\end{itemize}

\section{Example}

The following code snippet gives a valid example of an LPE:

\begin{lstlisting}
//Process definition:
PROCDEF example[A :: Int, B](state, curr, prev :: Int)
  = A ? i [[state==0]] >-> example[A, B](2, i, prev)
  + A ? i [[state==1 && i!=prev]] >-> example[A, B](2, i, prev)
  + A ? i [[state==2 && i==curr]] >-> example[A, B](1, curr, curr)
  + B >-> STOP
  ;

//Initialization:
example[A, B](0, 0, 0);
\end{lstlisting}

The process only accepts input sequences in which every number is repeated exactly once.
The process terminates non-deterministically.
