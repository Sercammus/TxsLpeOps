%% Based on a TeXnicCenter-Template by Gyorgy SZEIDL.
%%%%%%%%%%%%%%%%%%%%%%%%%%%%%%%%%%%%%%%%%%%%%%%%%%%%%%%%%%%%%

%----------------------------------------------------------
%
\documentclass{report}%
%
%----------------------------------------------------------
% This is a sample document for the standard LaTeX Report Class
% Class options
%       --  Body text point size:
%                        10pt (default), 11pt, 12pt
%       --  Paper size:  letterpaper (8.5x11 inch, default)
%                        a4paper, a5paper, b5paper,
%                       legalpaper, executivepaper
%       --  Orientation (portrait is the default):
%                       landscape
%       --  Printside:  oneside (default), twoside
%       --  Quality:    final (default), draft
%       --  Title page: titlepage, notitlepage
%       --  Columns:    onecolumn (default), twocolumn
%       --  Start chapter on left:
%                       openright(no), openany (default)
%       --  Equation numbering (equation numbers on right is the default)
%                       leqno
%       --  Displayed equations (centered is the default)
%                       fleqn (flush left)
%       --  Open bibliography style (closed bibliography is the default)
%                       openbib
% For instance the command
%          \documentclass[a4paper,12p,leqno]{report}
% ensures that the paper size is a4, fonts are typeset at the size 12p
% and the equation numbers are on the left side.
%
\usepackage{amsmath}%
\usepackage{amsfonts}%
\usepackage{amssymb}%
\usepackage{graphicx}
%----------------------------------------------------------
\usepackage{url}
\usepackage[square,sort,comma,numbers]{natbib}
\usepackage{chapterbib}
\usepackage{hyperref}
\usepackage{url}
\usepackage{doi}
\usepackage{listings}
\usepackage{color}
%----------------------------------------------------------
% Define colors:
\definecolor{dkgreen}{rgb}{0,0.6,0}
\definecolor{gray}{rgb}{0.5,0.5,0.5}
\definecolor{mauve}{rgb}{0.58,0,0.82}
% Define language
\lstdefinelanguage{torxakis}
{
  % list of keywords
  morekeywords={
    PROCDEF,
    FUNCDEF
  },
  sensitive=false, % keywords are not case-sensitive
  morecomment=[l]{//}, % l is for line comment
  morecomment=[s]{/*}{*/}, % s is for start and end delimiter
  morestring=[b]" % defines that strings are enclosed in double quotes
}
% Define environment:
\lstset{frame=tb,
  language=torxakis,
  aboveskip=3mm,
  belowskip=3mm,
  showstringspaces=false,
  columns=flexible,
  basicstyle={\small\ttfamily},
  numbers=none,
  numberstyle=\tiny\color{gray},
  keywordstyle=\color{blue},
  commentstyle=\color{dkgreen},
  stringstyle=\color{mauve},
  breaklines=true,
  breakatwhitespace=true,
  tabsize=3
}
%----------------------------------------------------------
\newcommand{\txs}{TorXakis}
\newcommand{\inlinecode}[1]{{\lstinline[language=torxakis]{#1}}}
%----------------------------------------------------------
\newtheorem{theorem}{Theorem}
\newtheorem{acknowledgement}[theorem]{Acknowledgement}
\newtheorem{algorithm}[theorem]{Algorithm}
\newtheorem{axiom}[theorem]{Axiom}
\newtheorem{case}[theorem]{Case}
\newtheorem{claim}[theorem]{Claim}
\newtheorem{conclusion}[theorem]{Conclusion}
\newtheorem{condition}[theorem]{Condition}
\newtheorem{conjecture}[theorem]{Conjecture}
\newtheorem{corollary}[theorem]{Corollary}
\newtheorem{criterion}[theorem]{Criterion}
\newtheorem{definition}[theorem]{Definition}
\newtheorem{example}[theorem]{Example}
\newtheorem{exercise}[theorem]{Exercise}
\newtheorem{lemma}[theorem]{Lemma}
\newtheorem{notation}[theorem]{Notation}
\newtheorem{problem}[theorem]{Problem}
\newtheorem{proposition}[theorem]{Proposition}
\newtheorem{remark}[theorem]{Remark}
\newtheorem{solution}[theorem]{Solution}
\newtheorem{summary}[theorem]{Summary}
\newenvironment{proof}[1][Proof]{\textbf{#1.} }{\ \rule{0.5em}{0.5em}}
%----------------------------------------------------------
\begin{document}

\title{LPE operations}
\author{Djurre van der Wal}
\date{\today}
\maketitle
\tableofcontents

\chapter{LPE}

\section{Introduction}
The techniques that are described in the following chapters must be applied to a \emph{linear process equation}, or LPE for short.

\section{Requirements}

An LPE must satisfy the following requirements:

\begin{itemize}

\item Given are the definition of a single process \emph{and} its instantiation.
The instantiation must be a closed expression, and otherwise satisfy the usual requirements of \txs{} of process instantiations.
There are no requirements for the channel parameters and data parameters beyond the usual requirements of \txs{} of process definitions.

\item The body of the process consists of one or more \emph{summands}.
In this context, a summand must contain exactly the following, in the given order:

\begin{itemize}

\item One or more \emph{channel communications}; that is, channel references followed by the variables that are used to communicate over those channels.
For example, \inlinecode{OUTPUT ? x ? y}, or simply \inlinecode{OUTPUT}.

Channel communications are combined with \inlinecode{|}.
To give another example, \inlinecode{INPUT ? x | OUTPUT ? y}.

The variables that are used to communicate over a channel must be \emph{fresh}.

\item A guard, yielding a boolean value.
The guard is not always explicitly written if it is semantically equivalent to \inlinecode{true}.

\item A sequence operator, \inlinecode{>->}.

\item The expression for deadlock (\inlinecode{STOP}) \emph{or} a recursive process instantiation.
The instantiated process must use the process definition given by the LPE, which satisfies the requirements of a \txs{} process signature.
This includes channel parameters: these must be assigned their current values of the instantiating process.
There are no requirements for the values assigned to the data parameters beyond the usual requirements of \txs{} (such as type compatibility).

\end{itemize}

\end{itemize}

\section{Example}

The following code snippet gives a valid example of an LPE:

\begin{lstlisting}
//Process definition:
PROCDEF example[A :: Int, B](state, curr, prev :: Int)
  = A ? i [[state==0]] >-> example[A, B](2, i, prev)
  + A ? i [[state==1 && i!=prev]] >-> example[A, B](2, i, prev)
  + A ? i [[state==2 && i==curr]] >-> example[A, B](1, curr, curr)
  + B >-> STOP
  ;

//Initialization:
example[A, B](0, 0, 0);
\end{lstlisting}

The process only accepts input sequences in which every number is repeated exactly once.
The process terminates non-deterministically.

\chapter{constElm}

\section{Introduction}

A linearized process may have parameters that do not change value throughout the state space of that process.
These parameters are essentially \emph{constants}.

Constants can be removed from a process as follows:
\begin{itemize}
\item Substitute references to a constant by the explicit value of that constant;
\item Remove the constant from the process parameter list.
\end{itemize}

The advantages of removing constants from a process are smaller state vectors (and therefore reduced memory usage) and faster performance in general.
These advantages may be small; however, the detection and removal of constants can be done fairly cheaply.

\section{Algorithm}

The algorithm consists of the following steps \cite{groote2001computer}:

\begin{enumerate}

\item Mark all process parameters.
(This means that, initially, we assume that all process parameters are constants.)

\item Let $P$ be the set of all marked process parameters.
Define a substitution $\rho = [p \rightarrow v_0(p) \;|\; p \in P]$, where $v_0$ is a function that gives the initial value of a given process parameter.

\item Consider each summand $s$ of the LPE.
Construct an equation $c_s \rightarrow p = v_s(p)$ for all $p \in P$, where $c_s$ is the guard of $s$ and where $v_s$ is a function that gives the expression of which the value is assigned to the process parameter $p$ in the instantiation, and apply the substitution $\rho$ to it.
(This gives $(c_s \rightarrow p = v_s(p))[\rho] \Leftrightarrow {c_s}[\rho] \rightarrow v_0(p) = v_s(p)[\rho]$.)

If the obtained equation is a tautology (that is, if its negation is unsatisfiable) for all $s$, $p$ remains marked; otherwise, unmark $p$.

\item Repeat the previous two steps until $P$ remains constant.
All remaining marked process parameters can be safely removed the process.

\end{enumerate}

\section{Example}

Consider the following LPE:

\begin{lstlisting}
//Process definition:
PROCDEF example[A](x, y, z :: Int)
  = A [[z = 2]] >-> example[A](z-1, 1, 2)
  + A >-> example[A](y, x, x+y)
  + A >-> example[A](1, x, z+1)
  ;

//Initialization:
example[A](1, 1, 2);
\end{lstlisting}

First, $\rho = [ x \rightarrow 1, y \rightarrow 1, z \rightarrow 2 ]$.

We must check the following equations:

\begin{align*}
(z = 2 \rightarrow x = z-1)[\rho] &\Leftrightarrow (2 = 2 \rightarrow 1 = 2-1) \Leftrightarrow \textit{true} \\
(z = 2 \rightarrow y = 1)[\rho] &\Leftrightarrow (2 = 2 \rightarrow 1 = 1) \Leftrightarrow \textit{true} \\
(z = 2 \rightarrow z = 2)[\rho] &\Leftrightarrow (2 = 2 \rightarrow 2 = 2) \Leftrightarrow \textit{true} \\
(x = y)[\rho] &\Leftrightarrow (1 = 1) \Leftrightarrow \textit{true} \\
(y = x)[\rho] &\Leftrightarrow (1 = 1) \Leftrightarrow \textit{true} \\
(z = x+y)[\rho] &\Leftrightarrow (2 = 1+1) \Leftrightarrow \textit{true} \\
(x = 1)[\rho] &\Leftrightarrow (1 = 1) \Leftrightarrow \textit{true} \\
(y = x)[\rho] &\Leftrightarrow (1 = 1) \Leftrightarrow \textit{true} \\
(z = z+1)[\rho] &\Leftrightarrow (2 = 2+1) \Leftrightarrow \textit{false} \\
\end{align*}

The last equation is not a tautology, and so $z$ is unmarked.

The new value of $\rho$ is $[ x \rightarrow 1, y \rightarrow 1 ]$.

\clearpage
The equations are now the following:

\begin{align*}
(z = 2 \rightarrow x = z-1)[\rho] &\Leftrightarrow (z = 2 \rightarrow 1 = z-1) \Leftrightarrow \textit{true} \\
(z = 2 \rightarrow y = 1)[\rho] &\Leftrightarrow (z = 2 \rightarrow 1 = 1) \Leftrightarrow \textit{true} \\
(x = y)[\rho] &\Leftrightarrow (1 = 1) \Leftrightarrow \textit{true} \\
(y = x)[\rho] &\Leftrightarrow (1 = 1) \Leftrightarrow \textit{true} \\
(x = 1)[\rho] &\Leftrightarrow (1 = 1) \Leftrightarrow \textit{true} \\
(y = x)[\rho] &\Leftrightarrow (1 = 1) \Leftrightarrow \textit{true} \\
\end{align*}

All of the equations above are tautologies, and so $x$ and $y$ remain marked.
Removing the marked parameters from the LPE gives

\begin{lstlisting}
//Process definition:
PROCDEF example[A](z :: Int)
  = A >-> example[A](2)
  + A >-> example[A](2)
  + A >-> example[A](z+1)
  ;

//Initialization:
example[A](2);
\end{lstlisting}

Obviously, more simplification is possible.


\chapter{parElm}

\section{Introduction}

A linearized process may have parameters that do not affect the behavior of that process in any way.
These parameters are called \emph{inert}.

Removing inert parameters from a process has the advantages of fewer states, smaller state vectors (and therefore reduced memory usage), and faster performance in general.
These advantages may be small; however, detection and removal of inert parameters can be done fairly cheaply.

\section{Algorithm}

The algorithm consists of the following steps \cite{groote2001computer}:

\begin{enumerate}

\item Mark all process parameters.
(This means that, initially, we assume that all process parameters are inert.)

\item Unmark all process parameters that occur in the guard of a summand.

\item Consider the assignments that occur as part of the recursive process instantiations of the summands of the LPE.
Unmark all process parameters that occur in the expression of which the value is assigned to an \emph{unmarked} process parameter.

\item Repeat the previous step until no process parameter is unmarked.
All remaining marked process parameters can be safely removed the process.

\end{enumerate}

\clearpage
\section{Example}

Consider the following LPE:

\begin{lstlisting}
//Process definition:
PROCDEF example[A :: Int, B](x, y, z :: Int)
  = A ? i [[x==0]] >-> example[A, B](i, y, z)
  + A ? i [[x==1]] >-> example[A, B](0, i, z)
  + B [[x==2]] >-> example[A, B](0, y, z)
  + B >-> example[A, B](z, y, x)
  ;

//Initialization:
example[A, B](0, 0, 0);
\end{lstlisting}

First, $x$ is unmarked, because it occurs in the guards of the first three summands.

In the fourth summand, $z$ is used in the expression of which the value is assigned to $x$.
Therefore $z$ must also be unmarked.

$y$ remains marked: it does not occur in a guard, nor is it used in the assignment to a process parameter other than itself.
Removing $y$ gives

\begin{lstlisting}
//Process definition:
PROCDEF example[A :: Int, B](x, z :: Int)
  = A ? i [[x==0]] >-> example[A, B](i, z)
  + A ? i [[x==1]] >-> example[A, B](0, z)
  + B [[x==2]] >-> example[A, B](0, z)
  + B >-> example[A, B](z, x)
  ;

//Initialization:
example[A, B](0, 0);
\end{lstlisting}




\chapter{parReset}

\section{Introduction}

A linearized process may have parameters that are initialized, changed and used, and subsequently ignored until they are initialized again.
Parameters that often follow this pattern are \emph{control flow parameters}.

During the period in between the last change or use of a parameter and its subsequent initialization, a parameter could have different values.
These values contribute to the size of the state space of the process \emph{without} adding any new behavior!
It may therefore be advantageous to detect from which moment the value of a parameter is no longer used and set it to a default value instead.

\section{Algorithm}

The algorithm is a generalization of an existing algorithm \cite{van2009state}.
It consists of two phases, a preparation phase and an iteration phase.

\subsection{Preparation phase}

Consider all possible pairs of summands of the LPE (including symmetric pairs).
Of a given summand pair $(s, t)$, let $t$ be a \emph{successor} of $s$ if $s$ contains a recursive process instantiation and if the following equation is satisfiable:

\begin{align*}
c_s \land {c_t}[p \rightarrow v_s(p) \;|\; p \in P]
\end{align*}

where

\begin{itemize}
\item $c_s$ is the guard of summand $s$;
\item $c_t$ is the guard of summand $t$;
\item $P$ is the set of all process parameters;
\item $v_s$ is a function that yields the expression that summand $s$ assigns to a given process parameter in its recursive process instantiation.
\end{itemize}

During the preparation phase, we determine all successors of each summand of the LPE.
(This gives an overapproximation of the control flow graph of the LPE.)

\subsection{Iteration phase}

This phase follows these steps:

\begin{enumerate}

\item For each summand $s$, create a set $R_s$ that contains all process parameters.
This means that, initially, we assume that all process parameters are used by one or more of the successors of $s$.

\item For each summand $s$, set the value of $R_s$ to $\bigcup\limits_{t \in S_s}^{} r(t)$ where $S_s$ is the set of all successors of $s$ (as determined during the previous phase) and where $r$ is the function

\begin{align*}
r(t) = P \cap \left( \text{vars}(c_t) \cup \bigcup\limits_{x \in R_t}^{} v_t(x) \right)
\end{align*}

where

\begin{itemize}
\item $P$ is the set of all process parameters;
\item $\text{vars}(c_t)$ gives the free variables in $c_t$, the guard of summand $t$;
\item $v_t$ is a function that yields the expression that summand $t$ assigns to a given process parameter in its recursive process instantiation.
\end{itemize}

\item Repeat the previous step until the new value of $R_s$ is the same as before for all summands $s$.

\item For each summand $s$ and each process parameter $x \notin R_s$, change the recursive process instantiation of $s$ so that $x$ is assigned ${v_s}[x \rightarrow x_0]$ for a choice of $x_0$ such that ${c_s}[x \rightarrow x_0]$ is satisfiable.
(This ensures that the set of summands that are successors of $s$ can only decrease.)

\end{enumerate}

\clearpage
\section{Example}

Consider the following LPE:

\begin{lstlisting}
//Process definition:
PROCDEF example[A :: Int, B](x, y :: Int)
  = A ? i [[x==0]] >-> example[A](1, i)
  + A ? i [[x==1 && i==y]] >-> example[A](2, y)
  + B [[x==2]] >-> example[A](3, y)
  + B [[x==3]] >-> example[A](0, y)
  ;

//Initialization:
example[A, B](0, 0);
\end{lstlisting}

Finding the successors of each summand is easy: each summand has exactly one successor, namely the next one, except in case of the fourth summand, where the first summand is the successor.

It is also obvious that $x$ will always be in $R_s$ for each summand $s$, because each summand uses $x$ in its guard.

Process parameter $y$ will always be in $R_{s_1}$, where $s_1$ is the first summand, because $y$ is used in the guard of $s_1$'s successor (the second summand).
After a few iterations, however, $y$ is removed from $R_{s_2}$ to $R_{s_4}$.
This means that $y$ is assigned a default value in the corresponding summands.
Depending on the mood of the SMT solver, this could give

\begin{lstlisting}
//Process definition:
PROCDEF example[A :: Int, B](x, y :: Int)
  = A ? i [[x==0]] >-> example[A](1, i)
  + A ? i [[x==1 && i==y]] >-> example[A](2, 0)
  + B [[x==2]] >-> example[A](3, 0)
  + B [[x==3]] >-> example[A](0, 0)
  ;

//Initialization:
example[A, B](0, 0);
\end{lstlisting}

Note that it is not required for the default value of $y$ to be consistent!



\bibliographystyle{plainnat}
\bibliography{biblio}

\end{document}
