\chapter{constElm}

\section{Introduction}

A linearized process may have parameters that do not change value throughout the state space of that process.
These parameters are essentially \emph{constants}.

Constants can be removed from a process as follows:
\begin{itemize}
\item Substitute references to a constant by the explicit value of that constant;
\item Remove the constant from the process parameter list.
\end{itemize}

The advantages of removing constants from a process are smaller state vectors (and therefore reduced memory usage) and faster performance in general.
These advantages may be small; however, the detection and removal of constants can be done fairly cheaply.

\section{Algorithm}

The algorithm consists of the following steps \cite{groote2001computer}:

\begin{enumerate}

\item Mark all process parameters.
(This means that, initially, we assume that all process parameters are constants.)

\item Define a substitution $\rho = [p \rightarrow v_0(p) \;|\; p \in P]$, where $P$ is the set of all process parameters and where $v_0$ is a function that gives the initial value of a given process parameter.

\item Consider each recursive process instantiations of the summands of the LPE.
Construct an equation $p \neq v(p)$ for all $p \in P$, where $v$ is a function that gives the expression of which the value is assigned to the process parameter $p$ in the instantiation, and applying the substitution $\rho$ to it.
If the result is unsatisfiable, $p$ remains marked, because $(p = v(p))[\rho] \Leftrightarrow v_0(p) = v(p)[\rho]$ seems to be an invariant (for now).
Otherwise, unmark $p$.

\item Repeat the previous step until no process parameter is unmarked.
All remaining marked process parameters can be safely removed the process.

\end{enumerate}

\section{Example}

Consider the following LPE:

\begin{lstlisting}
//Process definition:
PROCDEF example[A :: Int](x, y, z :: Int)
  = A >-> example[A](z-1, 1, 2)
  + A >-> example[A](y, x, x+y)
  + A >-> example[A](1, x, z+1)
  ;

//Initialization:
example[A](1, 1, 2);
\end{lstlisting}

First, $\rho = [ x \rightarrow 1, y \rightarrow 1, z \rightarrow 2 ]$.

We must check the satisfiability of the following equations:

\begin{align*}
(x \neq z-1)[\rho] &\Leftrightarrow (1 \neq 2-1) \Leftrightarrow \textit{false} \\
(y \neq 1)[\rho] &\Leftrightarrow (1 \neq 1) \Leftrightarrow \textit{false} \\
(z \neq 2)[\rho] &\Leftrightarrow (2 \neq 2) \Leftrightarrow \textit{false} \\
(x \neq y)[\rho] &\Leftrightarrow (1 \neq 1) \Leftrightarrow \textit{false} \\
(y \neq x)[\rho] &\Leftrightarrow (1 \neq 1) \Leftrightarrow \textit{false} \\
(z \neq x+y)[\rho] &\Leftrightarrow (2 \neq 1+1) \Leftrightarrow \textit{false} \\
(x \neq 1)[\rho] &\Leftrightarrow (1 \neq 1) \Leftrightarrow \textit{false} \\
(y \neq x)[\rho] &\Leftrightarrow (1 \neq 1) \Leftrightarrow \textit{false} \\
(z \neq z+1)[\rho] &\Leftrightarrow (2 \neq 2+1) \Leftrightarrow \textit{true} \\
\end{align*}

The last equation is clearly satisfiable, and so $z=2$ cannot be an invariant.
$z$ is unmarked, and $\rho$ becomes $[ x \rightarrow 1, y \rightarrow 1 ]$.

\clearpage
The equations are now the following:

\begin{align*}
(x \neq z-1)[\rho] &\Leftrightarrow (1 \neq z-1) \\
(y \neq 1)[\rho] &\Leftrightarrow (1 \neq 1) \Leftrightarrow \textit{false} \\
(x \neq y)[\rho] &\Leftrightarrow (1 \neq 1) \Leftrightarrow \textit{false} \\
(y \neq x)[\rho] &\Leftrightarrow (1 \neq 1) \Leftrightarrow \textit{false} \\
(x \neq 1)[\rho] &\Leftrightarrow (1 \neq 1) \Leftrightarrow \textit{false} \\
(y \neq x)[\rho] &\Leftrightarrow (1 \neq 1) \Leftrightarrow \textit{false} \\
\end{align*}

The first equation is unsatisfiable (take $z=0$), and so $x$ remains marked.
None of the other equations result in a parameter being unmarked, either, which means that $x$ and $y$ are constants.
Removing them from the LPE gives

\begin{lstlisting}
//Process definition:
PROCDEF example[A :: Int](z :: Int)
  = A >-> example[A](2)
  + A >-> example[A](2)
  + A >-> example[A](z+1)
  ;

//Initialization:
example[A](2);
\end{lstlisting}

Obviously, more simplification is possible.

