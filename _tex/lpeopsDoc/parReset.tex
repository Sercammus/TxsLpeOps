\chapter{parReset}

\section{Introduction}

A linearized process may have parameters that are initialized, changed and used, and subsequently ignored until they are initialized again.
Parameters that often follow this pattern are \emph{control flow parameters}.

During the period in between the last change or use of a parameter and its subsequent initialization, a parameter could have different values.
These values contribute to the size of the state space of the process \emph{without} adding any new behavior!
It may therefore be advantageous to detect from which moment the value of a parameter is no longer used and set it to a default value instead.

\section{Algorithm}

The algorithm is a generalization of an existing algorithm \cite{van2009state}.
It consists of two phases, a preparation phase and an iteration phase.

\subsection{Preparation phase}

Consider all possible pairs of summands of the LPE (including symmetric pairs).
Of a given summand pair $(s, t)$, let $t$ be a \emph{successor} of $s$ if $s$ contains a recursive process instantiation and if the following equation is satisfiable:
\begin{align*}
c_s \land {c_t}[p \rightarrow v_s(p) \;|\; p \in P]
\end{align*}

where

\begin{itemize}
\item $c_s$ and $c_t$ are the guards of summands $s$ and $t$, respectively;
\item $P$ is the set of all process parameters;
\item $v_s$ is a function that yields the expression that summand $s$ assigns to a given process parameter in its recursive process instantiation.
\end{itemize}

During the preparation phase, we determine all successors of each summand of the LPE.
This gives an overapproximation of the control flow graph of the LPE.

\subsection{Iteration phase}

This phase follows these steps:

\begin{enumerate}

\item For each summand $s$, create a set $R_s$ that contains all process parameters.
This means that, initially, we assume that all process parameters are used by one or more of the successors of $s$.

\item For each summand $s$, set the value of $R_s$ to $\bigcup\limits_{t \in S_s}^{} r(t)$ where $S_s$ is the set of all successors of $s$ (as determined during the previous phase) and where $r$ is the function
\begin{align*}
r(t) = \left( \text{vars}(c_t) \cup \bigcup\limits_{x \in R_t}^{} v_t(x) \right) \setminus C_t
\end{align*}

where

\begin{itemize}
\item $\text{vars}(c_t)$ gives the free variables in $c_t$, the guard of summand $t$;
\item $v_t$ is a function that yields the expression that summand $t$ assigns to a given process parameter in its recursive process instantiation;
\item $C_t$ is the set of the communication variables used by summand $t$.
\end{itemize}

\item Repeat the previous step until the new value of $R_s$ is the same as before for all summands $s$.

\item For each summand $s$ with $R_s \subset P$, choose some $\rho = [x \rightarrow h_x \;|\; x \in P \cup C_s]$ so that ${c_s}[\rho]$ is satisfiable, where

\begin{itemize}
\item $P$ is the set of all process parameters of the LPE;
\item $h_x$ is a closed expression of the same sort as $x$;
\item $C_t$ is the set of the communication variables used by summand $t$;
\item $c_s$ is the guard of summand $s$.
\end{itemize}

For each process parameter $x \in P \setminus R_s$, change the recursive process instantiation of $s$ so that $x$ is assigned ${v_s}(x)[\rho]$.
Note that this ensures that the set of summands that are successors of $s$ can only decrease.

Also note that it may be possible to merge summands in specific cases if the default values of process parameters are chosen cleverly and reused for multiple summands.
To find the optimal solution, however, a combinatorial problem must be solved, which is not expected to have a payoff that is proportional to the required effort.

\end{enumerate}

\section{Example}

Consider the following LPE:

\begin{lstlisting}
//Process definition:
PROCDEF example[A :: Int, B](x, y :: Int)
  = A ? i [[x==0]] >-> example[A](1, i)
  + A ? i [[x==1 && i==y]] >-> example[A](2, y)
  + B [[x==2]] >-> example[A](3, y)
  + B [[x==3]] >-> example[A](0, y)
  ;

//Initialization:
example[A, B](0, 0);
\end{lstlisting}

Finding the successors of each summand is easy: each summand has exactly one successor, namely the next one, except in case of the fourth summand, where the first summand is the successor.

It is also obvious that $x$ will always be in $R_s$ for each summand $s$, because each summand uses $x$ in its guard.

Process parameter $y$ will always be in $R_{s_1}$, where $s_1$ is the first summand, because $y$ is used in the guard of $s_1$'s successor (the second summand).
After a few iterations, however, $y$ is removed from $R_{s_2}$ to $R_{s_4}$.
This means that $y$ is assigned a default value in the corresponding summands.
Depending on the mood of the SMT solver, this could give

\begin{lstlisting}
//Process definition:
PROCDEF example[A :: Int, B](x, y :: Int)
  = A ? i [[x==0]] >-> example[A](1, i)
  + A ? i [[x==1 && i==y]] >-> example[A](2, 0)
  + B [[x==2]] >-> example[A](3, 0)
  + B [[x==3]] >-> example[A](0, 0)
  ;

//Initialization:
example[A, B](0, 0);
\end{lstlisting}




